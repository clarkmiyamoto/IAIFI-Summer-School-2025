%%% Document Formatting
\documentclass[12pt,fleqn]{article}
\usepackage[a4paper,
            bindingoffset=0.2in,
            left=0.75in,
            right=0.75in,
            top=0.8in,
            bottom=0.8in,
            footskip=.25in]{geometry}
\setlength\parindent{10pt} % No indent

%%% Imports
% Mathematics
\usepackage{amsmath} % Math formatting
\numberwithin{equation}{section} % Number equation per section
\DeclareMathOperator{\Tr}{Tr}


\newtheorem{theorem}{Theorem}
\newtheorem{definition}{Definition}
\newtheorem{proof}{Proof}
\newtheorem{lemma}{Lemma}
\newtheorem{example}{Example}

\usepackage{amsmath}
\usepackage{amsfonts} % Math fonts
\usepackage{amssymb} % Math symbols
\usepackage{mathtools} % Math etc.
\usepackage{slashed} % Dirac slash notation
\usepackage{cancel} % Cancels to zero
\usepackage{empheq}
\usepackage{breqn}

\newcommand{\expect}[1]{\mathbb{E}\left[#1\right]}


% Visualization
\usepackage{graphicx} % for including images
\graphicspath{ {} } % Path to graphics folder
\usepackage{tikz}



%%% Formating
\usepackage{hyperref} % Hyperlinks
\hypersetup{
    colorlinks=true,
    linkcolor=blue,
    filecolor=magenta,      
    urlcolor=cyan,
    pdftitle={Overleaf Example},
    pdfpagemode=FullScreen,
    }
\urlstyle{same}

\usepackage{mdframed} % Framed Enviroments
\newmdenv[ 
  topline=false,
  bottomline=false,
  skipabove=\topsep,
  skipbelow=\topsep
]{sidework} %% Side-work

\usepackage{lipsum} % Lorem Ipsum example text

%%%%% ------------------ %%%%%
%%% Title
\title{Notes from IAIFI Summer School 2025}
\author{Clark Miyamoto (cm6627@nyu.edu)}
\date{Date}
\begin{document}

\maketitle
\begin{abstract}
	IAIFI is a physics \& artificial intelligence research institution at MiT \& Harvard. Materials can be found at \url{https://github.com/iaifi/summer-school-2025}.
\end{abstract}
\tableofcontents



\newpage
\part{Sasha Rakhlin: Elements of Interactive Decision Making, Reinforcement Learning:}
This lecture is based off the course notes at \url{https://arxiv.org/abs/2312.16730}.

\section{Overview}
At each time $t$. The user gets some context $x_t$, makes a decision $\pi_t$. Through this decision, you get some reward $r_t$, and make an observation $o_t$. This is the high-level problem setup.

\subsection{Unstructured Multi-Armed Bandits (MAB)}
Consider a decision space $\Pi = [1, ..., \mathcal A]$,
For $t \in [1,..., T]$, the decision maker selects $\pi_t \sim p_t \in \Delta(\Pi)$ (a probability distribution over the decision space), and then observed reward $r_t = f^*(\pi_t) + \epsilon_t$.

We assume that $f^* : \Delta(\pi) \to  [0,1]^{\mathcal A}$ (is some bounded mean reward), and $\epsilon_t$ is (sub) Gaussian. Our objective function is regret
\begin{align}
	\text{Reg}_{\text{DM}} = \sum_{t=1}^T \mathbb E_{\pi_t} [f^*(\pi^*) - f^*(\pi_t)]
\end{align}
The question becomes, how can we make a choice for $\pi_t$ s.t. we grow the regret function sub-linearly in $T$. (If regret grows linearly in $T$, you're not learning!)
\begin{sidework}
\textbf{Greedy Choices Fail:}\\
Consider making an estimate for $f^*$ (denoted by $\hat f_t(\pi)$)
\begin{align}
	\hat f_t(\pi) = \frac{1}{|\tau_t(\pi)|} \sum_{s \in \tau_t(\pi)} r_s
\end{align}
where $\tau_t(\pi)=$ time steps prior to $t$ when $\pi$ was chosen. \\The greedy choice is
\begin{align}
	\pi_t = \text{argmax}_{\pi \in \Pi} \hat f_t(\pi)
\end{align}
This obviously fails, i.e. think of eating a restaurant once, and never revisiting it.
\end{sidework}
\begin{sidework}
	\textbf{Upper-Confidence-Bound (UCB):}\\
	At time $t$ choose the arm w/ largest $\text{ucb}_t(\pi)$
	\begin{align}
		\text{ucb}_t(\pi) = \hat f_t (\pi) + \underbrace{\sqrt{\frac{2\log \delta^{-1}}{|\tau_t(\pi)|}}}_{\text{Due to tails of } \epsilon}
	\end{align}
	and $\hat f_t(\pi) = \frac{1}{|\tau_t(\pi)} \sum_{s \in \tau_t(\pi)} r_s$, $\tau_t(\pi) = $ timesteps prior to $t$ when $\pi$ was chosen, $1-\delta$ is the confidence interval.\\
	\\
	What's nice is that the confidence interval bounds the error of your estimate of reward.\\
	\\
	You can show that 
	\begin{align}
		\text{Reg}_{\text{DM}} = \mathcal O(\sqrt{\mathcal A T \log (1/\delta)}
	\end{align}
\end{sidework}
These have been deterministic solutions, but what if you can use stochastic / randomized solutions?
\begin{sidework}
	\textbf{SquareCB Algorithm:}\\
	On each round $t \in [1, ..., T]$. Estimate $\hat f_t$ via regression w.r.t. data $\{(\pi_i, r_i)\}_{i=1}^{t-1}$. Compute the Inverse Gap weighting (IGW) distribution
	\begin{align}
		p_t(\pi) = \frac{1}{\lambda + \gamma (\hat f_t(\hat \pi) - \hat f_t(\pi)}
	\end{align}
	with $\lambda$ s.t. $\sum_\pi p_t(\pi) = 1$. Sample a decision $\pi_t \sim p$ and observe reward $r_t$. $\gamma$ is your temperature.\\
	\\
	You get a nearly optimal regret
	\begin{align}
		\text{Reg}_{\text{DM}} = \mathcal O(\mathcal A \sqrt{T})
	\end{align}
	You can also show that that when you sample your action dependent on IGW, risk is upper bounded by
	\begin{align}
		\text{Risk} \leq \frac{\mathcal A}{\gamma} + \gamma ~\mathbb E_{\pi \sim p} [(\hat f(\pi) - f^*(\pi))^2]
	\end{align}
\end{sidework}

\subsection{Structured Bandits}
Now consider where there are no contexts $x_t$ and observations $o_t$. This means for $t \in [1,...,T]$: the decision-maker selects $\pi_t \sim p_t \in \Delta(\Pi)$ (where $\Pi$ is possibly infinite now) and observes reward $r_t = f^*(\pi_t) + \epsilon$. \\
\\
We assume that a known model class $\mathcal F$ s.t. $f^* \in \mathcal F$ (the true mean reward is inside a function class, typically defined by a neural network), $\epsilon_t$ is (sub) Gaussian.\\
\\
Our objectives are regret, or risk.\\
\\
Recall in the MAB setting, we estimated rewards for each arm $\pi \in \Pi$. We now have a possibly infinite set $\Pi$, so what is the analogue of mean estimate if we know the true model is in $\mathcal F$? This is regression (best fit within $\mathcal F$ to past data $\{(\pi_i, r_i)\}_{i=1}^{t-1}$
\begin{align}
	\text{argmin}_{f\in \mathcal F} \sum_{s < t} (f(\pi_s) - r_s)^2
\end{align}
\begin{sidework}
	Define a confidence set $\mathcal F_t \subseteq \mathcal F$. Define upper and lower confidence functions s.t.
	\begin{align}
		\text{lcb}_t(\pi) = \min_{f \in \mathcal F_t} f(\pi)\\
		\text{ucb}_t(\pi) = \max_{f \in \mathcal F_t}f(\pi)
	\end{align}
	If you make the optimistic choice $\pi_t = \text{argmax
}_{\pi \in \Pi} \text{ucb}_t(\pi)$,	you can show that regret is upper bounded by
	\begin{align}
		\text{Reg}_{\text{DM}}\leq \sum_{t=1}^T (\text{ucb}_t(\pi) - \text{lcb}_t(\pi))
	\end{align}
\end{sidework}

\section{Contextual Bandits}
Consider the problem for $t = \in[1,...,T]$. The decision maker observes $x_t \in \mathcal X$, the decision maker selects $\pi_t \sim p_T \in \Delta(\Pi)$, and observes reward $r_t  = f^*(x_t, \pi_t) + \epsilon_t$.\\
\\
We assume a known model class $\mathcal F$, s.t. $f^* \in \mathcal F$. \\
\\
The objective is either regret and risk.
\begin{align}
	\text{Reg} = \sum_t \mathbb E_{\pi_t} [f^*(x_t, \pi^*) - f^*(x_t, \pi_t)]
\end{align}
\begin{sidework}
	\textbf{SquareCB Algorithm:} You can get the previous SquareCB Algorithm, and lift $f(\pi) \mapsto f(x,\pi)$. Plug and chug.
\end{sidework}
\begin{sidework}
	\textbf{LinearUCB:} There's also an extension of the linear UCB algorithm.
\end{sidework}

\subsection{Episodic RL}
Consider a state space $\mathcal S$, and action space $\mathcal A$. You have a probability transition kernel $P_h^M : \mathcal S \times \mathcal A\to \Delta(S)$, and a reward distribution $P_h^M: \mathcal S \times \mathcal A \to \Delta(\mathbb R)$, and initalize your state according to distribution $d_1 \in \Delta(S)$. You have dynamics for episode $t \in [1,...,T]$: for $h \in [1,..,H]$, you take action $a_h \sim \pi_h(\cdot | s_h)$ according to policy $\pi$, this yields a reward $r_h \sim \mathcal R_h^M(\cdot | s_h, a_h)$, and finally you observe a new state (due to your action) $s_{h+1} \sim P_h^M(\cdot | s_h , a_h)$.


































\newpage
\part{Gaia Grosso: Physics-Motivated Optimization}

\newpage
\part{SueYeon Chung: Representation/Manifold Learning, Geometric Deep Learning}

\newpage
\part{Aleksandra \'Ciprajanovi\'c: Domain Shift Problem: Building Robust AI Models with Domain Adaptation}










\end{document}